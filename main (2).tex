\documentclass[10.5pt]{ctexart}
\usepackage{titlesec}
\usepackage{graphicx}
\usepackage{indentfirst}
\usepackage{hyperref}

\setlength\parindent{2em}

\title{命令行环境与Python}
\author{张中瑞}
\date{9月19日}

\begin{document}
\maketitle

\section{练习内容}
\subsection{命令行环境练习内容}
\begin{enumerate}
    \item 进程管理操作:使用pgrep和pkill命令查找并结束进程
    使用wait命令在一个进程结束后再执行另一个进程,编写pidwaitbash 函数,持续等待指定 PID 进程结束,使用sleep减少 CPU 消耗
    \item 终端多路复用:tmux的使用和自定义tmux的配置
    \item 命令别名设置:创建dc别名替代错误输入的dc
    \item 远程设备的连接:ssh密钥配对;
    配置.ssh文件设置虚拟机连接内容;
    使用sssh-copy-id copy ssh密钥到服务器;
    使用虚拟机Web服务器的本地端口转发访问;
    配置ssh服务器,禁用密码验证和root登录。
    \end{enumerate}
\subsection{python入门基础练习内容}
\begin{enumerate} 
    \item 变量赋值与打印:name = "小明"; print(name) → 定义字符串变量并输出内容
    \item 整数计算与输出:print(5 + 3 * 2) → 计算表达式并打印结果(输出 11)
    \item 字符串拼接:print("Hello " + "Python") → 合并两个字符串(输出 Hello Python)
    \item 列表创建与取值:fruits = ["苹果", "香蕉"]; print(fruits[0]) → 定义列表并取第一个元素(输出苹果)
    \item 简单条件判断:age = 18; print("成年" if age >= 18 else "未成年") → 按条件输出结果(输出成年)
    \item 循环打印数字:for i in range(3): print(i) → 循环输出 0、1、2
    \item 基础函数定义:def add(a,b): return a+b; print(add(2,3)) → 定义加法函数并调用(输出 5)
    \item 读取用户输入:user_input = input("请输入姓名:"); print(user_input) → 获取用户输入并打印
    \item 字典取值:info = {"age":20}; print(info["age"]) → 定义字典并获取指定键的值(输出 20)
    \item 浮点数格式化:print(f"身高:{1.75:.1f}米") → 保留 1 位小数输出(输出身高:1.8 米)
\end{enumerate}
\subsection{python视觉应用练习内容}
\begin{enumerate}
    \item 图片的读写与信息查看:用Image.open()打开,img.size获取尺寸,img.save()保存。
    \item 图片缩放与裁剪:img.resize()缩放,img.crop()裁剪。
    \item 图片旋转与翻转:例如将图片顺时针旋转90°,img.rotate(90)用来顺时针旋转
    \item 灰度图与黑白图转换:img.convert('L')转灰度,用point()方法实现阈值处理。
    \item 图片的亮度:亮度为原来的1.5倍,adjustedimg = enhancer.enhance(1.5)。
    \item 图片的对比度处理:adjustedimg = contrastenhancer.enhance(factor=1.5),factor=1.0,保持原对比度;
    \item 图片的模糊处理:blurredimg = img.filter(ImageFilter.GaussianBlur(radius=5)),eadius越大越模糊
    \item 图片的锐化处理:sharpened =img.filter(ImageFilter.SHARPEN)
\end{enumerate}

\section{练习结果}
\subsection{命令行环境练习}
\begin{itemize}
\item 进程管理:通过pgrep -af "进程名"可精准定位 PID,pkill -f无需手动输入 PID 即可结束进程;wait命令确保子进程执行完毕后再运行后续命令,pidwait函数利用kill -0检测进程存活状态,配合sleep避免 CPU 占用过高。
\item tmux:掌握窗口拆分、会话保存等操作,自定义.tmux.conf可提升多任务处理效率(如快捷键映射)。
\end{itemize}

\subsection{Python 基础练习}
\begin{itemize}
\item 变量与运算:理解数据类型(字符串、数值、布尔)的赋值与转换,如int("123")可将字符串转整数。
\item 流程控制:if-else条件判断和for/while循环是逻辑核心,例如用for i in range(10)遍历 0-9。
\item 数据结构:列表(list)、字典(dict)的索引与取值操作,如dict["key"]获取对应值。
\end{itemize}

\subsection{Python 视觉应用练习}
\begin{itemize}
\item 图片基础操作:用Image.open()读取图片后,可通过size属性获取宽高,save()指定格式(如 PNG)。
\item 特效处理:亮度 / 对比度调整通过ImageEnhance模块实现,factor>1增强效果;高斯模糊radius=5可柔化图片,锐化SHARPEN滤镜突出边缘。
\item 格式转换:灰度图convert('L')丢失色彩信息,黑白图通过point()阈值处理(如x>128设为白色)。
\end{itemize}

\section{解题感悟}
\begin{enumerate}
    \item 命令行:掌握进程管理与远程连接是高效运维的基础,tmux 等工具能大幅提升多任务处理能力。
    \item Python 语法:基础数据类型与流程控制是核心,需大量的练习。
    \item 图像处理:Pillow 库封装了底层操作,理解像素、通道等概念后,可灵活实现特效(如去噪、锐化),为进阶(如 OpenCV 计算机视觉)打下基础。
\end{enumerate}

\end{document}




